% ET722A Luciano Albuquerque Lima Saraiva
\documentclass[12pt,a4paper]{article}
\usepackage[utf8]{inputenc}
\usepackage[onehalfspacing]{setspace}
\usepackage[lmargin=3cm,tmargin=3cm,rmargin=2cm,bmargin=2cm]{geometry}
\usepackage[brazil]{babel}
\usepackage[T1]{fontenc}
\usepackage{graphicx}
\usepackage{indentfirst}
\usepackage{comment}
\usepackage{enumerate}
\usepackage{hyperref}
\usepackage{tabularx}
\usepackage{subfig}
\usepackage{glossaries}
\usepackage{amssymb}

% numeracao header
\pagestyle{headings}

% quebra de pagina por secao
\let\oldsection\section 
\renewcommand\section{\clearpage\oldsection}


\begin{document}

\begin{titlepage}
	\begin{center}
		\textbf{UNIVERSIDADE DE RIBEIRÃO PRETO} \\
			CIÊNCIAS EXATAS \\
			Engenharia de Software I

			\vspace{1.5cm}

			\textbf{Prof. Luciano A. L. Saraiva}
			\vspace{0.5cm}

			\textbf{Murilo da Silva Ijanc'} \\
			RA: 834125

			\vspace{6.5cm}

			\textbf{ATIVIDADE FINAL} \\
			\textbf{CINEMA SARAIVA}

			\vfill

			\vspace{0.8cm}

			\includegraphics[width=0.2\textwidth]{unaerp}

			\textbf{RIBEIRÃO PRETO} \\
			\textbf{2020}
	\end{center}
\end{titlepage}


\thispagestyle{empty}

\tableofcontents

\pagenumbering{arabic}

% secao 1
\section{Missão}
O software visa solucionar a gestão dos assentos do \textbf{Cinema Saraiva}, bem
como a elaboração de um relatório estatístico contendo informações relevantes para
o proprietário tomada de decisão. O produto deve rodar em plataformas Unix da família
BSD e com consumo de memória inferior a 16MB. A empresa se responsabiliza pelo
desenvolvimento do software e agendamento de entrevista com o cliente, já o cliente
necessita ter disponibilidade para as entrevistas que irão ser necessárias para o 
desenvolvimento do software.

\section{Entrevista}
Primeiro a empresa entrou em contato com o cliente (01/07/2020) para um teste para ver se o
mesmo recebia os emails, já que devido a pandemia atual não poderia realizar
reuniões física.

Após confirmado o recebimento do email de teste, foi levantado algumas questões
para extrair informações que ajudariam no desenvolvimento das funcionalidades do 
sistema e depois foram enviadas (03/07/2020), porêm as respostas não foram recebidas 
por nossos servidores de email como pode ser visto no log abaixo.\\


\textbf{Primeiro contato}:


maillog.6.gz:Jul  1 00:58:56 lina smtpd[69372]: d289b56cd956a220 mta delivery evpid=74b2a2178a949108 from=<mbsd@m0x.ru> to=<llsaraiva@unaerp.br> rcpt=<-> source="46.23.94.11" relay="189.112.110.65 (mx01.unaerp.br)" delay=33m17s result="Ok" stat="250 2.6.0 message received"\\
maillog.6.gz:Jul  1 01:20:01 lina smtpd[69372]: d289b574ed5dce6d smtp envelope evpid=87437c145382497f from=<llsaraiva@unaerp.br> to=<mbsd@m0x.ru>\\
maillog.6.gz:Jul  1 01:20:01 lina smtpd[69372]: d289b575edaf635a mda delivery evpid=87437c145382497f from=<llsaraiva@unaerp.br> to=<mbsd@m0x.ru> rcpt=<mbsd@m0x.ru> user=mbsd delay=3s result=Ok stat=Delivered\\


\textbf{Segundo contato}:


maillog.4.gz:Jul  3 00:25:51 lina smtpd[69372]: d289b9ea2bbbc3c6 mta delivery evpid=1450d3d36a614eb1 from=<mbsd@m0x.ru> to=<llsaraiva@unaerp.br> rcpt=<-> source="46.23.94.11" relay="189.112.110.65 (mx01.unaerp.br)" delay=11s result="Ok" stat="250 2.6.0 message received"\\


Com a necessidade de entrega de outros projetos antes desse foi priorizado essas
entregas e por tanto as funciondalidades destacadas no decorrer desse projeto
foram extrapoladas pela empresa contrante afim de pelo menos ter as funcionalidades
básicas do sistema.

\section{Requisitos funcionais}
\subsection{[RF001] Escolha do filme}

\textbf{Prioridade}:
\mbox{\ooalign{$\checkmark$\cr\hidewidth$\square$\hidewidth\cr}} Essencial
\mbox{\ooalign{\cr\hidewidth$\square$\hidewidth\cr}} Importante
\mbox{\ooalign{\cr\hidewidth$\square$\hidewidth\cr}} Desejável

O sistema deve permitir que o cliente do estabelecimento escolha os filmes disponíveis
para a sala.

\subsection{[RF002] Reserva de lugares}

\textbf{Prioridade}:
\mbox{\ooalign{$\checkmark$\cr\hidewidth$\square$\hidewidth\cr}} Essencial
\mbox{\ooalign{\cr\hidewidth$\square$\hidewidth\cr}} Importante
\mbox{\ooalign{\cr\hidewidth$\square$\hidewidth\cr}} Desejável

O sistema deve permitir caso o assento esteja disponível o cliente selecioná-lo,
a quantidade de assentos que o cliente pode selecionar é até o número máximo de
assentos disponíveis da sala.


\subsection{[RF003] Preço ingresso}

\textbf{Prioridade}:
\mbox{\ooalign{$\checkmark$\cr\hidewidth$\square$\hidewidth\cr}} Essencial
\mbox{\ooalign{\cr\hidewidth$\square$\hidewidth\cr}} Importante
\mbox{\ooalign{\cr\hidewidth$\square$\hidewidth\cr}} Desejável

O sistema deve permitir o proprietário a adicionar o preço do valor do
ingresso inteira, automaticamente o sistema deve dividir o preço da inteira
pela metade computando assim o preço do ingresso inteiro e meia entrada.

\subsection{[RF004] Tipo de entrada}

\textbf{Prioridade}:
\mbox{\ooalign{$\checkmark$\cr\hidewidth$\square$\hidewidth\cr}} Essencial
\mbox{\ooalign{\cr\hidewidth$\square$\hidewidth\cr}} Importante
\mbox{\ooalign{\cr\hidewidth$\square$\hidewidth\cr}} Desejável

O sistema deve permitir que o cliente selecione meia ou entrada inteira.

\subsection{[RF005] Visualização dos assentos da sala}

\textbf{Prioridade}:
\mbox{\ooalign{$\checkmark$\cr\hidewidth$\square$\hidewidth\cr}} Essencial
\mbox{\ooalign{\cr\hidewidth$\square$\hidewidth\cr}} Importante
\mbox{\ooalign{\cr\hidewidth$\square$\hidewidth\cr}} Desejável

O sistema deve desenhar os assentos da sala em forma de uma matriz, cada
linha identificada por uma letra do alfabeto A-Z e as colunas por números de
1-9.

\subsection{[RF005] Visualização dos assentos da sala}

\textbf{Prioridade}:
\mbox{\ooalign{$\checkmark$\cr\hidewidth$\square$\hidewidth\cr}} Essencial
\mbox{\ooalign{\cr\hidewidth$\square$\hidewidth\cr}} Importante
\mbox{\ooalign{\cr\hidewidth$\square$\hidewidth\cr}} Desejável

O sistema deve desenhar os assentos da sala em forma de uma matriz, cada
linha identificada por uma letra do alfabeto A-Z e as colunas por números de
1-9.

\subsection{[RF006] Número de ingressos meia/inteira por sala}

\textbf{Prioridade}:
\mbox{\ooalign{$\checkmark$\cr\hidewidth$\square$\hidewidth\cr}} Essencial
\mbox{\ooalign{\cr\hidewidth$\square$\hidewidth\cr}} Importante
\mbox{\ooalign{\cr\hidewidth$\square$\hidewidth\cr}} Desejável

O sistema deve calcular o número de ingressos meia entrada e inteira por sala.

\subsection{[RF007] Número total de ingressos por sala}

\textbf{Prioridade}:
\mbox{\ooalign{$\checkmark$\cr\hidewidth$\square$\hidewidth\cr}} Essencial
\mbox{\ooalign{\cr\hidewidth$\square$\hidewidth\cr}} Importante
\mbox{\ooalign{\cr\hidewidth$\square$\hidewidth\cr}} Desejável

O sistema deve calcular o número total de ingressos por sala.

\subsection{[RF008] Fatura por sala}

\textbf{Prioridade}:
\mbox{\ooalign{$\checkmark$\cr\hidewidth$\square$\hidewidth\cr}} Essencial
\mbox{\ooalign{\cr\hidewidth$\square$\hidewidth\cr}} Importante
\mbox{\ooalign{\cr\hidewidth$\square$\hidewidth\cr}} Desejável

O sistema deve calcular o total faturado por sala.

\subsection{[RF008] Número de lugares disponíveis por sala}

\textbf{Prioridade}:
\mbox{\ooalign{$\checkmark$\cr\hidewidth$\square$\hidewidth\cr}} Essencial
\mbox{\ooalign{\cr\hidewidth$\square$\hidewidth\cr}} Importante
\mbox{\ooalign{\cr\hidewidth$\square$\hidewidth\cr}} Desejável

O sistema deve calcular o número de lugares disponíveis por sala.

\subsection{[RF008] Checagem de idade}

\textbf{Prioridade}:
\mbox{\ooalign{$\checkmark$\cr\hidewidth$\square$\hidewidth\cr}} Essencial
\mbox{\ooalign{\cr\hidewidth$\square$\hidewidth\cr}} Importante
\mbox{\ooalign{\cr\hidewidth$\square$\hidewidth\cr}} Desejável

O sistema deve receber o ano de nascimento do cliente quando o mesmo
solicitar 

\section{Requisitos de Sistema}

\subsection{[RS001] Rodar plataforma UNIX}

\textbf{Prioridade}:
\mbox{\ooalign{\cr\hidewidth$\square$\hidewidth\cr}} Essencial
\mbox{\ooalign{$\checkmark$\cr\hidewidth$\square$\hidewidth\cr}} Importante
\mbox{\ooalign{\cr\hidewidth$\square$\hidewidth\cr}} Desejável

O sistema deve rodar em plataformas UNIX da família \textbf{BSD}.

\subsection{[RS002] Suporte teclado físico}

\textbf{Prioridade}:
\mbox{\ooalign{$\checkmark$\cr\hidewidth$\square$\hidewidth\cr}} Essencial
\mbox{\ooalign{\cr\hidewidth$\square$\hidewidth\cr}} Importante
\mbox{\ooalign{\cr\hidewidth$\square$\hidewidth\cr}} Desejável

O sistema deve conseguir ler a entrada de dados pelo teclado físico.

\subsection{[RS003] Salvar o status assentos}

\textbf{Prioridade}:
\mbox{\ooalign{$\checkmark$\cr\hidewidth$\square$\hidewidth\cr}} Essencial
\mbox{\ooalign{\cr\hidewidth$\square$\hidewidth\cr}} Importante
\mbox{\ooalign{\cr\hidewidth$\square$\hidewidth\cr}} Desejável

O sistema deve conseguir recuperar os dados dos assentos em caso de falha, como
por exemplo queda de energia.

\section{Ciclo}

A metodologia para o desenvolvimento foi \textbf{Cascata} por já saber
quais requisitos eram necessário para geração do relatório.

\subsection{Levantamento de requisitos}
Apesar de não ter finalizado a comunicação para o levantamento de todos
os requisitos, durante a primeira reunião já se obteve algumas funcionalidades
necessárias para o funcionamento básico do sistema.

\subsection{Planejamento}
No planejamento definiu-se o tempo e o custo para o desenvolvimento dos
requisitos do sistema bem como seriam acompanhado a implementação dessas
funcionalidades.

\subsection{Modelagem}
Durante a modelagem foram decidos aspectos primordias quanto a aquitetura
do sistema e a interface, por exemplo como seriam apresentados os assentos
para o cliente. Como o proprietário quis diminuir os custos utilizando software
livre o sistema teria que rodar na plataforma do mesmo.

\subsection{Construção}
O sistema foi desenvolvido na linguagem C, para manter mais fidedigno possível
os desenvolvedores utilizaram em suas máquinas o mesmo sistema operacional que 
roda nas máquinas do Cinema. Um script feito em shell é utilizado para testes
unitários das funcionalidades internas do sistema, quando as funcionalidades
externas aguardamos a execução pelo cliente.


\subsection{Implementação}
A implementação ainda será realiza após o testes realizados pelo cliente. Como
foi decidido para automatiza o processo de implementação foi criado um Makefile
para ajudar a instalar o sistema nas máquinas do Cinema.


\section{Q.A}


\end{document}
